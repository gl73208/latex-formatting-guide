% ЕСЛИ ССЫЛКИ НЕ ОТОБРАЖАЮТСЯ ИЛИ НЕ ОБНОВИЛИСЬ: При компиляции документа информация об имеющихся метках добавляется в файл с расширением aux. Чтобы извлечь эту информацию, (номер раздела или рисунка - команда \ref) , необходимо пропустить текст через latex ещё раз.
\documentclass[a4paper, times new roman, oneside]{article}% Определение класса документа, задание шрифта, параметров страницы


% Работа со шрифтами
\usepackage[T2A]{fontenc} % Загрузка стандартного пакета для выбора внутренней кодировки LaTeX
\usepackage[utf8]{inputenc} % Загрузка стандартного пакета для указания, в какой кодировке набран текст
\usepackage[english, russian]{babel} % Загрузка языковых пакетов (английский, русский шрифты)
\usepackage[14pt]{extsizes} % Загрузка набора размеров шрифтов, выбор основного размера шрифта
\usepackage[usenames]{color} % Пакет для выделения текста цветом


% Работа с формулами
\usepackage{amsmath} % Загрузка пакетов для работы с формулами


% Работа с гипертекстом
\usepackage{hyperref} % Получение гипертекстового документа
\usepackage{totpages} % Пакет предоставляет информацию о числе страниц, получившихся в результате компиляции 


% Работа с титульным листом
\title{\vspace{4cm} \Huge Форматирование \\ в системе \LaTeX \\ \vspace{1cm}  \huge Пример файла \\ \vspace{1cm} \Large Листов \pageref{TotPages}}% Заголовок, используется при печати титульной страницы командой \maketitle
\author{} % Фамилия, имя автора (не указаны)
\date{} % Дата (не проставлена)


% Работа с оглавлением
\usepackage[titles]{tocloft}% Загрузка пакета для оформления оглавления
\renewcommand{\cftsecleader}{\cftdotfill{\cftdotsep}}% Настройка содержания


% Работа с таблицами
\usepackage{multirow}% Загрузка пакета для объединения ячеек таблицы 
\usepackage{array}% Загрузка пакета для выравнивания строк таблицы
\usepackage{makecell} % Загрузка пакета для создания многострочных ячеек в таблицах


% Работа с рисунками
\usepackage{graphicx} % Загрузка пакета для работы с рисунками
\graphicspath{{pictures/}}% Указали путь к каталогу с рисунками
\DeclareGraphicsExtensions{.png, .jpg} %Указали расширения файлов изображений
\usepackage{float}% Загрузка пакета для корректировки вертикальных отбивок для рисунков


% Работа с заголовками
\renewcommand{\thesection}{\arabic{section}.} % Добавили точку после нумерации заголовка
\renewcommand{\thesubsection}{\thesection\arabic{subsection}.}
\renewcommand{\thesubsubsection}{\thesubsection\arabic{subsubsection}.}
\usepackage{titlesec} % Загрузили альтернативный пакет для формирования заголовков разделов
\titleformat{\section}[hang]{\normalfont\large\bfseries}{\thesection}{1em}{} % То же для заголовков section
\titlespacing*{\section}{24pt}{3.25ex plus 1ex minus .2ex}{0.7em}
\titleformat{\subsection}[hang]{\normalfont\normalsize\bfseries}{\thesubsection}{1em}{} % То же для заголовков subsection
\titlespacing*{\subsection}{24pt}{3.25ex plus 1ex minus .2ex}{0.5em}
\titleformat{\subsubsection}[hang]{\normalfont\normalsize\bfseries}{\thesubsubsection}{1em}{} % То же для заголовков subsubsection
\titlespacing*{\subsubsection}{24pt}{3.25ex plus 1ex minus .2ex}{0.5em}
\newcommand{\sectionbreak}{\clearpage} % Задали начало раздела с новой страницы 


% Работа с подписями к таблицам и рисункам
\usepackage[figurename=Рисунок]{caption}% Загрузка пакета для работы с подписями к плавающим объектам
\DeclareCaptionLabelSeparator{dash}{ --- } % Определили тип разделителя
\captionsetup{labelsep=dash} % В названиях таблиц и рисунков заменили : на -


% Работа с многостраничными таблицами
\usepackage{longtable}% Загрузка пакета для многостраничных таблиц 
%\usepackage{ltcaption} % Загрузка пакета для подписи для многостраничных таблиц
\captionsetup{skip=5pt} % Отбивка для подписи



% Работа со списками
\renewcommand{\theenumi}{\asbuk{enumi}}% Отображение счетчика первого уровня строчными буквами кириллицы
\renewcommand{\labelenumi}{\asbuk{enumi}) }% В счетчике первого уровня заменили а. на а)
\renewcommand{\theenumii}{\arabic{enumii}}% Отображение счетчика второго уровня с помощью арабских цифр 
\renewcommand{\labelenumii}{\arabic{enumii}) }% В счетчике второго уровня заменили (1) на 1)


% Работа с приложением
\usepackage{appendix}% Загрузка пакета для работы с приложением
\newcommand{\intro}[1]{             % Определение новой команды intro
    \stepcounter{section}   % Декларация \stepcounter изменяет значения существующего счетчика section
    \section*{\hfill\it{ПРИЛОЖЕНИЕ} \arabic{section}}\vspace{-5pt}\hfill{\itСправочное}\hfil\\% Команда секционирования начинает новый раздел, озаглавленный как Приложение
    \begin{center}
        \bf{#1} % Сюда помещается замещающий текст
    \end{center}
    \addcontentsline{toc}{section}{Приложение \arabic{section}. #1} % Добавляем названия разделов приложения к оглавлению
}


% Работа с макетом документа
\usepackage{vmargin} % Загрузка пакета для управления размерами макета через выставку полей
\usepackage{layout} % Загрузка пакета для вывода информации о текущем макете                                                                      
\hypersetup{pdfstartview=FitH} % Вид страницы при открытии pdf-файла. FitH - выравнивание по ширине. bookmarks=true по умолчанию (из оглавления формируются закладки) 
\setmarginsrb{20mm}{12mm}{10mm}{15mm}{14pt}{10mm}{0pt}{0mm} % Установка значений полей (левое, верхнее, правое, нижнее, высота верхнего колонтитула, расстояние между верхним колонтитулом и текстом,  высота нижнего колонтитула, расстояние между текстом и нижним колонтитулом


% Работа с колонтитулами
\usepackage{fancyhdr} % Загрузка пакета для работы с колонтитулами
\pagestyle{empty} % Очищаем стиль страницы: нет ни колонтитулов, ни номеров страниц 
\pagestyle{fancy} % Включаем пользовательский стиль для колонтитулов 
\fancyhead{} % Очистили верхний колонтитул
\fancyhead[CO]{\small \arabic{page}\\\vspace{3pt} Пример \LaTeX \,файла} % Задаем колонтитул
\fancyfoot{} % Очистили нижний колонтитул
\renewcommand{\headrulewidth}{0pt} % Убрали линию между верхним колонтитулом и текстом 


% Работа с абзацами
\usepackage{indentfirst} % Формирование отступа (красной строки) у первого параграфа (для формирования красной строки у остальных абзацев необходимо перед абзацем оставлять пустую строку)
\usepackage{parskip} % Загрузка пакета для задания параметров абзаца
\setlength\parskip{10pt} % Вертикальный промежуток между абзацами
\setlength\parindent{24pt} % Значение абзацного отступа (красной строки)
\renewcommand{\baselinestretch}{1.25} % Межстрочный интервал 1,5 интервала (1.25 - для шрифта 14)
\clubpenalty=10000 % Запрет разрыва абзаца после первой строки
\widowpenalty=10000 % Запрет висячих строк



\begin{document} % Ниже идет текст документа, выше находится преамбула
%\layout % Если убрать знак комментария и отключить команду \maketitle, на первой странице будет напечатан макет документа

\maketitle % Титульный лист
\thispagestyle{empty} % Очищаем стиль страницы: нет ни колонтитулов, ни номеров страниц 

\newpage
\begin{center}

{\bf Аннотация}
\end{center}

Настоящий документ является примером \LaTeX \,файла. \LaTeX \,(произносится «латех») --- замечательная альтернатива программам {\it Microsoft Word}, {\it LibreOffice} и другим текстовым редакторам, поскольку:
\begin{itemize} % Ненумерованный список, начало
\renewcommand{\labelitemi}{\cdash--*} % Заменили значок, которым помечаются элементы перечня (черный кружок заменили на тире)
\item  относится к свободно распространяемому ПО; % Командой \item вводится каждый элемент перечня
\item предназначен для работы в любых операционных системах; при переносе текста с одного компьютера на другой форматирование сохраняется, даже если компьютеры работают под разными операционными системами;
\item позволяет автору сосредоточиться на содержании текста, принимая на себя заботы по его оформлению;
\item превосходит остальные издательские системы по качеству текстов, содержащих математические формулы.
\end{itemize} % Ненумерованный список, конец

\begin{sloppypar}
\LaTeX \,не является монолитной программой, а состоит из набора пакетов. Дистрибутивы и документация к \LaTeX \, находятся на страничке проекта: http://www.ctan.org.\par

Также существуют онлайн редакторы \LaTeX \,, например, {\it Overleaf} (https://ru.overleaf.com). В данном редакторе предусмотрена автоматическая подсказка и подстановка команд, что позволяет работать с ним уже на начальном этапе знакомства с \LaTeX.
\end{sloppypar}


\renewcommand\contentsname{\hfil \hbox to 2.3in {СОДЕРЖАНИЕ}} % Оформление стиля заголовка оглавления
\tableofcontents % Вывод оглавления








\section{Как происходит работа с \LaTeX} % Название раздела
\label{sec:work} % Метка для раздела 
Для начала автор готовит файл с текстом в любом редакторе, оснащенном командами \LaTeX \,(настоящий файл создан при помощи редактора \TeX \,Live).  Такие файлы имеют расширение \texttt{tex}. Далее файл необходимо обработать при помощи компилятора.\par

Исходный файл в самом начале должен содержать команду\par

\texttt{\textbackslash\/documentclass[опции]\{класс\}} \par

в которой \texttt{[опции]} является необязательным, а \texttt{\{класс\}} --- обязательным аргументом, содержащим название класса печатного документа:
\begin{itemize} % Ненумерованный список, начало
\item \texttt{article} --- статья или небольшой отчет;
\item \texttt{letter} --- письмо;
\item \texttt{report} --- статья, разбитая на главы или небольшая книга;
\item \texttt{book} --- книга;
\item \texttt{proc} --- текст в две колонки;
\item \texttt{slides} --- презентация.
\end{itemize}

Таким образом, команда \texttt{\textbackslash\/documentclass} определяет основную структуру печатного документа.

Следующей обязательной командой является\par

\texttt{\textbackslash\/begin\{document\}}\par

с которой начинается текст документа. Текст перед  \texttt{\textbackslash\/begin\{document\}} называется преамбулой и содержит команды настройки выбранного класса печатного документа, а также определения новых команд. Заканчивается текст документа командой\par

\texttt{\textbackslash\/end\{document\}}\par

\newpage
\subsection{Пример \TeX \,файла}\label{sec:work:example} % Название раздела и метка

Теперь можно создать простейший \TeX \,файл:

\texttt{\textbackslash\/documentclass\{article\} \\
\indent \textbackslash\/begin\{document\} \\
\indent Example \\
\indent \textbackslash\/end\{document\}}

После компиляции появится печатная версия файла, на которую будет выведен результат компиляции --- слово \texttt{Example}.

\subsection{Пакеты \LaTeX}\label{sec:work:package} % Название раздела и метка

Пакет --- это служебный текстовый файл, при помощи которого в документ вносятся дополнения и изменения. Пакеты загружаются в редактируемый документ путем добавления в преамбулу команды

\texttt{\textbackslash\/usepackage[опции]\{имя пакета\}}

Опции и имена пакетов, если их несколько, перечисляются через запятую.







\section{Набор текста}\label{sec:text}

\subsection{Русификация \LaTeX}\label{sec:text:russian} % Название подраздела и метка

Чтобы перейти на русскоязычный шрифт, необходимо подключить несколько пакетов:

\texttt{\textbackslash\/documentclass\{article\} \\
\indent \textbackslash\/usepackage[T2A]\{fontenc\} \\ 
\indent \textbackslash\/usepackage[utf8]\{inputenc\} \\
\indent \textbackslash\/usepackage[russian]\{babel\} \\
\indent \textbackslash\/begin\{document\} \\
\indent Пример \\
\indent \textbackslash\/end\{document\}}

В примере в исходный файл были включены:
\begin{itemize}
\item \texttt{fontenc} --- пакет для выбора внутренней кодировки шрифтов \LaTeX;
\item \texttt{inputenc} --- пакет для указания кодировки текста в исходном файле;
\item \texttt{babel} --- пакет поддержки языков.
\end{itemize}

\subsection{Шрифты}\label{sec:text:font} % Подраздел второго уровня

\subsubsection{Выделение текста}\label{sec:text:font:type} % Подраздел второго уровня
Для набора {\bf полужирного} текста используется команда \texttt{\textbackslash\/textbf\{текст\}} или декларация \texttt{\{\textbackslash\/bf текст\}}.

Для {\it курсивного} начертания используется команда \texttt{\textbackslash\/textit\{текст\}} или декларация \texttt{\{\textbackslash\/it текст\}}.

\begin{sloppypar}
Наконец, чтобы получить \texttt{моноширинный} шрифт, используется команда \texttt{\textbackslash\/texttt\{текст\}} или декларация \texttt{\{\textbackslash\/tt текст\}}.
\end{sloppypar}

\subsubsection{Размер шрифта}\label{sec:text:font:size} % Подраздел второго уровня

Декларации для переключения размера шрифта приведены в таблице~\ref{tab:fontsize}. Текст, размер которого необходимо изменить, помещается в фигурных скобках сразу после декларации.

\setlength{\extrarowheight}{5pt} % Добавлен вертикальный промежуток между текстом и верхней границей ячейки
\begin{longtable}[htb]{|m{5cm}<\centering|m{5.5cm}<\centering|m{6cm}<\centering|}
\captionsetup{singlelinecheck=false}
\caption*{\hbox to 3.7in {\tablename~\thetable{}~--- Декларации, переключающие размер шрифта}} % Название таблицы
\label{tab:fontsize}\\ %Метка
\hline
{\bf Декларация} & {\bf Название} & {\bf Образец} \\ % Заголовок 
\hline
\endfirsthead
\caption*{\hbox to 3.8in {Продолжение таблицы \thetable{} \hfill }}\\ % Название таблицы для отображения на последующих страницах
\hline
{\bf Декларация} & {\bf Название} & {\bf Образец} \\ 
\hline
\endhead
\texttt{\textbackslash\/tiny}   & крошечный & \tiny{Аа \ldots яЯ} \\  \hline
\texttt{\textbackslash\/scriptsize}   & индексный & \scriptsize{Аа \ldots яЯ} \\ \hline
\texttt{\textbackslash\/footnotesize}   & подстрочный & \footnotesize{Аа \ldots яЯ} \\ \hline
\texttt{\textbackslash\/small}   & маленький & \small{Аа \ldots яЯ} \\ \hline
\texttt{\textbackslash\/normalsize}   & стандартный & \normalsize{Аа \ldots яЯ} \\ \hline
\texttt{\textbackslash\/large}   & большой & \large{Аа \ldots яЯ} \\ \hline
\texttt{\textbackslash\/Large}   & огромный & \Large{Аа \ldots яЯ} \\ \hline
\texttt{\textbackslash\/LARGE}   & громадный & \LARGE{Аа \ldots яЯ} \\ \hline
\texttt{\textbackslash\/huge}   & грандиозный & \huge{Аа \ldots яЯ} \\ \hline
\texttt{\textbackslash\/Huge}   & колоссальный & \Huge{Аа \ldots яЯ} \\ \hline
\end{longtable}

Чтобы задать основной размер шрифта, необходимо загрузить пакет {\tt extsizes}, содержащий набор размеров шрифтов и указать размер шрифта в необязательном аргументе:

{\tt \textbackslash\/usepackage[14pt]\{extsizes\}}







\subsection{Пробелы}\label{sec:text:space} % Подраздел второго уровня

Слова в тексте отделяются друг от друга пробелами, при этом не имеет значения, сколько пробелов набрано: \LaTeX \,автоматически преобразует их в один. Чтобы вручную управлять горизонтальными промежутками, есть специальные команды, описанные в таблице~\ref{tab:space}.\par

\setlength{\extrarowheight}{1pt}
\begin{longtable}[htb]{|m{3.5cm}<\centering|m{10cm}<\centering|m{3cm}<\centering|}
\captionsetup{singlelinecheck=false}
\caption*{\hbox to 3.7in {\tablename~\thetable{}~--- Команды, задающие пробел}} % Название таблицы
\label{tab:space}\\ %Метка
\hline
{\bfКоманда} & {\bf Вид пробела} & | | \\ % Заголовок 
\hline
\endfirsthead
\caption*{\hbox to 3.8in {Продолжение таблицы \thetable{} \hfill }}\\ % Название таблицы для отображения на последующих страницах
\hline
{\bfКоманда} & {\bf Вид пробела} & | | \\ 
\hline
\endhead
\texttt{\textbackslash\/quad}   & Пробел в 1em & |\quad| \\ \hline
\texttt{\textbackslash\/qquad} & Пробел в 2em& |\qquad| \\ \hline
\texttt{\textbackslash,} & «Тонкий пробел», или тонкая шпация & |\,| \\  \hline
\texttt{\textbackslash:} &«Средний пробел» & |\:| \\  \hline
\texttt{\textbackslash;} & «Толстый пробел»& |\;|  \\ \hline
\texttt{\textbackslash!} & «Отрицательный тонкий пробел» &\\  \hline
\texttt{\~{}}			  & Неразрывный пробел &\\  \hline
\texttt{\textbackslash/}   & Корректирующий пробел (используется, например, при переходе от курсива к прямому шрифту)  &\\  \hline
\texttt{\textbackslash\/hspace\{длина\}}  & Пробел заданной длины   &\\  \hline  
\texttt{\textbackslash\/hfill} & Пробел бесконечно растяжимой длины  &\\  \hline  
\end{longtable}







\subsection{Абзацы}\label{sec:text:indent}

\begin{sloppypar}
Абзацы разделяются между собой пустыми строками; чтобы сформировать абзац для первой строки, необходимо загрузить пакет \texttt{indentfirst} (команда \texttt{\textbackslash\/usepackage\{indentfirst\}} в преамбуле).\par
\end{sloppypar}

Если не найдена подходящая точка для переноса, текст может выйти за границу правого поля:

Здесь текст, который вышел за границу поля из-за команды: \texttt{\textbackslash\/usepackage\{indentfirst\}}.

В этом случае необходима дополнительная защита от переполнения строки:

\texttt{\textbackslash\/begin\{sloppypar\}\\
\indent текст абзаца\\
\indent \textbackslash\/end\{sloppypar\}}

В случае, когда необходимо подавить абзацный отступ, удобно воспользоваться командой \texttt{\textbackslash\/noindent}, которая располагается впереди текста абзаца:

\texttt{\textbackslash\/noindent текст абзаца}

Чтобы завершить текущую страницу и начать с новой страницы, используется команда {\tt \textbackslash\/newpage}.








\newpage
\subsection{Спецсимволы}\label{sec:text:symbol}

Следующие 10 символов:

\texttt{\{ \} \$ \& \# \% \_ \^{} \~{} \textbackslash}

являются специальными; чтобы получить первые семь в печатной версии файла, необходимо в исходном тексте поставить перед соответствующим символом знак \texttt{\textbackslash} без пробела. Символы \texttt{\^{}} и \texttt{\~{}} получаются из комбинаций \texttt{\textbackslash\^{}\{\}} и \texttt{\textbackslash\~{}\{\}} в исходном файле, а символ \texttt{\textbackslash} --- из команды \texttt{\textbackslash\/textbackslash}.

\subsection{Дефисы и тире}\label{sec:text:symbol}

Для получения дефиса (используется в сложных составных словах, например, \texttt{северо-запад}), в исходном тексте набирается один знак \texttt{-}.

Для получения короткого тире (используется при задании числовых промежутков, например, 5--15), набираются два знака - (\texttt{-\/-}).

Тире (---) используется в качестве знака препинания в предложениях и состоит из трех знаков \texttt{-} (\texttt{-\/-\/-}).

\subsection{Многоточие}\label{sec:text:dots}

Для получения многоточия (\ldots) используется команда \texttt{\textbackslash\/ldots}.

Для заполнения пустого пространства в строке точками используется команда \texttt{\textbackslash\/dotfill}: \dotfill

\subsection{Подчеркивание, рамки}\label{sec:text:box}

Чтобы \underline{подчеркнуть} слово, используется команда \texttt{\textbackslash\/underline\{подчеркиваемый текст\}}.

\begin{sloppypar}
Чтобы поместить текст в \fbox{рамку}, необходимо использовать команду \texttt{\textbackslash\/fbox\{текст\}}.
\end{sloppypar}








\section{Списки}\label{sec:list}

\subsection{Ненумерованный список}\label{sec:list:itemize}

Для составления ненумерованного списка используется процедура

\texttt{\textbackslash\/begin\{itemize\} \\
\indent \textbackslash\/item \\
\indent \textbackslash\/item \\
\indent \ldots \\
\indent \textbackslash\/end\{itemize\}}

\subsection{Нумерованный список}\label{sec:list:enumerate}

Для составления нумерованного списка используется процедура

\texttt{\textbackslash\/begin\{enumerate\}\\
\indent \textbackslash\/item\\
\indent \textbackslash\/item\\
\indent \ldots \\
\indent \textbackslash\/end\{enumerate\}}

По умолчанию нумерованный список является числовым. Чтобы преобразовать его в буквенный, необходимо указать в преамбуле (для всего документа) или непосредственно перед списком (команда действует до конца документа, если ее не перезаписать):

\texttt {\textbackslash\/renewcommand\{\textbackslash\/theenumi\}\{\textbackslash\/asbuk\{enumi\}\}}

Теперь, чтобы заменить точку после буквы на круглую скобку, необходимо добавить команду

\texttt{\textbackslash\/renewcommand\{\textbackslash\/labelenumi\}\{\textbackslash\/asbuk\{enumi\})\}}

\subsubsection{Как исключить букву из нумерованного буквенного списка}\label{sec:list:enumerate:letter}

Иногда требуется исключить из списка определенные цифры или буквы (например, по требованию ГОСТа необходимо исключать буквы «з» и «о» из русскоязычного буквенного списка). В этом случае необходимо:

\begin{enumerate}
\item начать список, как обычно, с команды \texttt{\textbackslash\/begin\{enumerate\}};
\item продолжить список;
\item \ldots
\item \ldots
\item \ldots
\item \ldots
\begin{sloppypar}
\item  $\leftarrow$ закончить список на букве, предшествующей исключаемой (команда \texttt{\textbackslash\/end\{enumerate\}});
\end{sloppypar}
\end{enumerate} %Разбили список на два, чтобы исключить из нумерации букву «з»
\begin{enumerate}
\setcounter{enumi}{8} %Новый список начнется с 8-й по счету - буквы «и»
\item начать новый список, дополнив его командой \texttt{\textbackslash\/setcounter\{enumi\}\{8\}}, позволяющей указать, с какого элемента начнется нумерация (в данном случае с восьмого --- буквы «и», так как в начале списка находится нулевой элемент);
\item таким образом, буква «з» исключена из буквенного нумерованного списка;
\item \ldots
\end{enumerate}
\ldots

\subsection{Многоуровневый список}\label{sec:list:multilevel}

Исходный код для ненумерованного многоуровневого списка выглядит следующим образом:

\texttt{\textbackslash\/begin\{itemize\} \\
\indent \textbackslash\/item \\
\indent \textbackslash\/item \\
\indent \hspace{1cm} \textbackslash\/begin\{itemize\} \\
\indent \hspace{1cm} \textbackslash\/item \\
\indent \hspace{1cm} \textbackslash\/item \\
\indent \hspace{1cm}  \ldots \\
\indent \hspace{1cm} \textbackslash\/end\{itemize\}\\
\indent \ldots \\
\indent \textbackslash\/end\{itemize\}}


\LaTeX \, допускает четыре уровня вложенности списка:


\begin{itemize}
\item Первый уровень 
\begin{itemize}
\item Второй уровень
\begin{itemize}
\item Третий уровень
\begin{itemize}
\item Четвертый (последний) уровень
\end{itemize}
\end{itemize}
\end{itemize}
\end{itemize}

Чтобы заменить метки, используемые процедурой {\tt itemize} по умолчанию, с помощью команды \texttt{\textbackslash\/renewcommand} переопределяются команды

\begin{tabular}{|c|}
\hline
{\tt \textbackslash\/labelitemi} \qquad {\tt \textbackslash\/labelitemii} \\
{\tt \textbackslash\/labelitemiii} \qquad {\tt \textbackslash\/labelitemiv} \\ \hline
\end{tabular}

\begin{sloppypar}

\vspace{10pt}
Например, команда

\texttt{\textbackslash\/renewcommand\{\textbackslash\/labelitemi\}\{\$\textbackslash\/circ\$\}}

заменяет элемент первого уровня на символ  бинарных операторов \texttt{$\circ$} (\texttt{\textbackslash\/circ}). Знаки \$ вокруг записи символа означают переход в математический режим форматирования, позволяющий использовать в тексте символы математических операций, а также формулы, о которых речь пойдет в следующем разделе.
\end{sloppypar}

Для нумерованного списка элементами второго уровня по умолчанию являются буквы английского алфавита, заключенные в круглые скобки. Их можно заменить на арабские цифры при помощи команды
 
\texttt{\textbackslash\/renewcommand\{\textbackslash\/theenumii\}\{\textbackslash\/arabic\{enumii\}\}}

Для замены круглых скобок на закрывающую круглую скобку используется команда

\texttt{\textbackslash\/renewcommand\{\textbackslash\/labelenumii\}\{\textbackslash\/arabic\{enumii\})\}}







\section{Формулы}\label{sec:formula}

Чтобы использовать в тексте формулы, необходимо подключить в преамбуле пакет \texttt{amsmath}:

\texttt{\textbackslash\/usepackage\{amsmath\}}

Для небольших формул, которые размещаются внутри абзаца, существует процедура форматирования математических формул \texttt{math} и три варианта обращения к ней:

\begin{tabular}{|c|}
\hline
\texttt{\textbackslash\/begin\{math\} \ldots \;\textbackslash\/end\{math\}} \\ 
\texttt{\textbackslash\/( \ldots \;\textbackslash\/)} \\ 
\texttt{\$ \ldots \;\$} \\ \hline
\end{tabular}


\vspace{10pt}
Процедура {\tt equation} производит формулы, расположенные в отдельной строке, и автоматически нумерует их. Существует два варианта обращения к данной процедуре:

\begin{tabular}{|c|}
\hline
\texttt{\textbackslash\/begin\{equation\} \ldots \;\textbackslash\/end\{equation\}} \\ 
\texttt{\$\$ \ldots \;\$\$} \\ \hline
\end{tabular}

Чтобы отключить нумерацию, необходимо процедуру \texttt{equation} заменить на \texttt{equation*}.

В таблице \ref{tab:formula} приведены команды для задания математических знаков, используемых в формулах.

\LTleft=0pt
\begin{longtable}[htb]{|m{10cm}<\centering|m{5cm}<\centering|}
\captionsetup{singlelinecheck=off}
\caption*{\hbox to 3.7in {\tablename~\thetable{}~--- Команды для задания математических знаков}} % Название таблицы
\label{tab:formula}\\ %Метка
\hline
{\bfКоманда} & {\bf Пример} \\ % Заголовок 
\hline
\endfirsthead
\caption*{\hbox to 3.8in {Продолжение таблицы \thetable{} \hfill }}\\ % Название таблицы для отображения на последующих страницах
\hline
{\bfКоманда} & {\bf Пример} \\ 
\hline
\endhead
\texttt{\textbackslash\/sqrt[степень]\{подкоренное выражение\}} & $\sqrt[3]{a+b}$  \\ \hline
\texttt{\textbackslash\/frac\{числитель\}\{знаменатель\}} & $\frac{a+b}{c}$ \\ \hline
\texttt{выражение\^{}степень} & $b^2$ \\  \hline
\texttt{\textbackslash\/cos} & $\cos x$ \\  \hline
\texttt{\textbackslash\/sin} & $\sin x$ \\ \hline
\texttt{\textbackslash\/tg} & $\tg x$ \\ \hline
\texttt{\textbackslash\/ctg} & $\ctg x$ \\ \hline
\texttt{\textbackslash\/exp} & $\exp$ \\ \hline
\texttt{\textbackslash\/ln\{выражение\}} & $\ln2$ \\ \hline
\texttt{\textbackslash\/log\_степень выражение} & $\log_2 10$ \\  \hline
\texttt{\textbackslash\/lim\_предел} & $\lim_{n\to\infty}$ \\  \hline
\texttt{\textbackslash\/min} & $\min$ \\ \hline
\texttt{\textbackslash\/max} & $\max$ \\ \hline
\texttt{\textbackslash\/sum\_\{нижний предел\}\^{}\{верхний предел\}} & $\sum_{i=1}^n$ \\ \hline
\texttt{\textbackslash\/int\_\{нижний предел\}\^{}\{верхний предел\}} & $\int_0^{\infty}$ \\ \hline
\texttt{ (\, \textbackslash\/big(\, \textbackslash\/Big(\, \textbackslash\/bigg(\, \textbackslash\/Bigg(} & $( \big( \Big( \bigg( \Bigg($ \\ \hline
\texttt{ [\, \textbackslash\/big[\, \textbackslash\/Big[\, \textbackslash\/bigg[\, \textbackslash\/Bigg[} & $[ \big[ \Big[ \bigg[ \Bigg[$ \\ \hline
\end{longtable}











\section{Заголовки}\label{sec:head}

Для класса документа вида {\tt article} существуют следующие команды секционирования:

\noindent{\tt \textbackslash\/section\{Название раздела\}}\\
{\tt \textbackslash\/subsection\{Название подраздела\}}\\
{\tt \textbackslash\/subsubsection\{Название пункта\}}\\
{\tt \textbackslash\/paragraph\{Название параграфа\}}

Каждая из приведенных выше команд начинает новую составную часть документа, то есть присваивает ей номер, печатает заголовок и вносит этот заголовок в оглавление.

Для печати оглавления используется команда {\tt \textbackslash\/tableofcontents}.













\section{Перекрестные ссылки}\label{sec:link}

Чтобы сослаться на раздел, рисунок, таблицу, формулу или пункт перечня, необходимо поставить метку:

{\tt \textbackslash\/label\{метка\}}
 
после команды именования соответствующего окружения. В случае таблиц или картинок такой командой является {\tt \textbackslash\/caption}, в случае пунктов перечня --- {\tt \textbackslash\/item}.

Метка представляет любую комбинацию латинских букв, цифр и некоторых знаков препинания. В частности при составлении меток удобно использовать двоеточие (:) и дефис (-).

Когда ссылки идут через метку, то номер раздела (команда {\tt \textbackslash\/ref}) и номер страницы (команда {\tt \textbackslash\/pageref}) определяется \LaTeX \,автоматически.

Например, для заголовка пункта \ref{sec:text:font:size} была поставлена метка 

{\tt \textbackslash\/label\{sec:text:font:size\}}

где через двоеточие указаны наименования раздела, подраздела и, наконец, пункта. Для ссылки на данный пункт используется команда

{\tt \textbackslash\/ref\{sec:text:font:size\}}










\section{Рисунки}\label{sec:picture}

Чтобы работать в \LaTeX \,с графическими изображениями, необходимо загрузить пакет {\tt graphicx} (см.~\ref{sec:work:package}). Также предварительно нужно создать каталог(и) для рисунков и указать путь в преамбуле:

\texttt {\textbackslash\/graphicspath\{\{каталог 1/\}\{каталог 2/\}\ldots\}} 

\begin{sloppypar}
Расширения графических файлов задаются при помощи команды {\tt \textbackslash\/DeclareGraphicsExtensions\{exts\}}, где аргумент {\tt exts} --- список расширений имен файлов, перечисленных через запятую, например

{\tt \textbackslash\/DeclareGraphicsExtensions\{.png, .jpg\}}

Чтобы разместить в тексте изображение, используется команда {\tt \textbackslash\/includegraphics\{\}}, где в фигурных скобках указывается имя файла. Ниже приведен пример использования данной команды:
\end{sloppypar}

{\bf Исходный код}

{\tt Кнопка \textbackslash\/includegraphics\{compile\} запускает процесс компиляции.}

{\bf Вывод на печать}

Кнопка \includegraphics{compile} запускает процесс компиляции.

\subsection{Процедура {\tt figure}}\label{sec:picture:figure}

Для автоматизированного размещения рисунков существует процедура {\tt figure}, рассматривающая рисунок как плавающий объект (если для объекта нет места на текущей странице, он переносится на следующую). Ниже приведен пример использования процедуры {\tt figure}:

\noindent {\tt \textbackslash\/begin\{figure\}[htb] \\ 
    \textbackslash\/centering  \textcolor{red}{\% Размещение по центру} \\
    \textbackslash\/includegraphics[width=\textbackslash\/textwidth]\{имя файла\}  \\ 
    \textbackslash\/caption\{Название рисунка\}  \textcolor{red}{\% Подрисуночная подпись} \\
    \textbackslash\/label\{fig:label\} \textcolor{red}{\% Метка} \\ 
\textbackslash\/end\{figure\}}

  Пример изображения представлен на рис.~\ref{fig:pic}.
  
\begin{figure}[htb] % Процедура для размещения рисунка
    \centering % Размещение по центру
    \includegraphics[width=\textwidth]{example} % Рисунок example.jpg загружен с помощью онлайн редактора изображений
    \vspace{-15pt}
    \caption{Пример изображения} % Подрисуночная подпись
    \label{fig:pic} % Метка
\end{figure}

Необязательный аргумент процедуры {\tt figure} задает способы размещения плавающего объекта:

\noindent {\tt h} --- разместить по возможности здесь же; \\
{\tt t} --- разместить в верхней части страницы; \\
{\tt b} --- разместить в нижней части страницы; \\
{\tt p} --- разместить на отдельной странице, где нет ничего кроме плавающих объектов.

Если указано несколько букв, возможен любой из предусматриваемых этими буквами вариантов размещения.

Команда {\tt \textbackslash\/caption\{\}} создает подпись к рисунку и печатает порядковый номер рисунка. Для работы с подписями к плавающим объектам необходимо загрузить пакет {\tt caption} (см.~\ref{sec:work:package}).

По умолчанию в \LaTeX \, подрисуночная подпись начинается со слова Рис., после которого следует двоеточие и название рисунка:

{\tt Рис.: Название рисунка}

Чтобы представить подрисуночную подпись в виде

{\tt Рисунок} --- {\tt Название рисунка}

необходимо в преамбуле добавить команды:

\noindent {\tt {\textbackslash\/DeclareCaptionLabelSeparator\{dash\}\{ -\/-\/-\ \}\\
\textbackslash\/captionsetup\{labelsep=dash\}}}

и добавить опцию при подключении пакета {\tt caption}:

\noindent {\tt {\textbackslash\/usepackage[figurename=Рисунок]\{caption\}}}

{\tt \textbackslash\/textwidth} --- это переменная, хранящая значение длины, равной ширине текста. Размерами рисунка также можно управлять через параметр {\tt scale}, например: 

{\tt \textbackslash\/includegraphics[scale=0.25]\{имя файла\}}







\section{Таблицы}\label{sec:table}

\subsection{Процедура {\tt tabular}}\label{sec:table:tabular}

Процедура \texttt{tabular} задает таблицу. В обязательном аргументе, помещаемом в фигурных скобках непосредственно после команды {\tt \textbackslash\/begin\{tabular\}}, задается последовательность букв (по одной на столбец), обозначающих выравнивание текста в ячейке:\\
{\tt l} --- выравнивание слева; \\
{\tt c} --- выравнивание по центру; \\
{\tt r} --- выравнивание справа.

{\tt p\{ширина\}} --- еще один способ задать столбец; означает абзац текста, ширина которого задается в фигурных скобках.

Символ {\tt \&} делит данные на ячейки. Разделительные линии между столбцами задаются с помощью вертикальной черты~{\tt |}. Горизонтальные линии создаются при помощи команды {\tt \textbackslash\/hline}. Строки в теле процедуры разделяются командой~\texttt{\textbackslash\/\textbackslash}. Заканчивается процедура командой {\tt \textbackslash\/end\{tabular\}}.\par


Ниже приведен исходный код для создания таблицы в \LaTeX:

\noindent\texttt{\textbackslash\/begin\{tabular\}\{|l|l|\}    \textcolor{red}{\% Два столбца, выравнивание слева}  \\
 	\textbackslash\/hline   \textcolor{red}{\% Горизонтальная линия} \\
 	\{\textbackslash\/bf Заголовок 1\} \& \{\textbackslash\/bf Заголовок 2\} \textbackslash\/\textbackslash \, \textcolor{red}{\% Заголовок} \\ 
 	\textbackslash\/hline \\
 	ячейка  1/1 	\& \, ячейка  1/2 \textbackslash\/\textbackslash \, \textbackslash\/hline \textcolor{red}{\% Тело таблицы} \\
 	ячейка  2/1	 \& \, ячейка  2/2 \textbackslash\/\textbackslash \, \textbackslash\/hline \\
 	ячейка  3/1	 \& \, ячейка  3/2 \textbackslash\/\textbackslash \, \textbackslash\/hline \\
 	\textbackslash\/end\{tabular\}}

В результате компиляции вышеприведенного кода получится таблица:

\begin{tabular}{|l|l|} % Описание для колонок (выравнивание слева)
\hline % Горизонтальная линия
{\bf Заголовок 1} & {\bf Заголовок 2} \\ \hline
ячейка 1/1 & ячейка 1/2 \\ \hline % Тело таблицы
ячейка 2/1 & ячейка 2/2 \\ \hline
ячейка 3/1 & ячейка 3/2 \\ \hline
\end{tabular}


\subsection{Процедура {\tt table}}\label{sec:table:table}

Процедура \texttt{table} служит для задания названия таблицы, метки для ссылки, параметров размещения таблицы относительно остального текста и автоматической нумерации.\par

\begin{sloppypar}
Для работы с подписью к таблице необходимо подключить пакет {\tt caption} (см.~\ref{sec:work:package}). Тогда команда {\tt \textbackslash\/captionsetup\{singlelinecheck=off\}} позволит выровнять по левому краю подпись, которая по умолчанию располагается в центре страницы; команды {\tt \textbackslash\/setlength\{\textbackslash\/abovecaptionskip\}\{\ldots\}} и {\tt \textbackslash\/setlength\{\textbackslash\/belowcaptionskip\}\{\ldots\}} позволят управлять вертикальными отбивками вокруг подписи.
\end{sloppypar}


\noindent\texttt{\textbackslash\/begin\{table\}[htb]  \textcolor{red}{\% Размещение таблицы} \\
\textbackslash\/captionsetup\{singlelinecheck=off\}  \textcolor{red}{\% Подпись выровнена по левому краю} \\
\textbackslash\/setlength\{\textbackslash\/abovecaptionskip\}\{0pt\}  \textcolor{red}{\% Отступ между названием и таблицей} \\
\textbackslash\/setlength\{\textbackslash\/belowcaptionskip\}\{5pt\}  \textcolor{red}{\% Отступ между текстом и названием} \\
\textbackslash\/caption\{Название таблицы\}   \textcolor{red}{\% Название таблицы} \\
\textbackslash\/label\{tab:label\}   \textcolor{red}{\% Метка} \\
\textbackslash\/begin\{tabular\} \\
\ldots \\
\textbackslash\/end\{tabular\} \\
 \textbackslash\/end\{table\} }

  
Теперь после компиляции таблица будет пронумерована, и к ней добавятся название и метка для ссылки (см. таблицу~\ref{tab:label}).

\begin{table}[htb] % Размещение таблицы
\captionsetup{singlelinecheck=off} % Подпись выровнена по левому краю
\setlength{\abovecaptionskip}{0pt} % Отступ между названием таблицы и таблицей
\setlength{\belowcaptionskip}{5pt} % Отступ между текстом и названием таблицы
\caption{Название таблицы}
\label{tab:label} % Метка
\begin{tabular}{|l|l|} % Описание для колонок (выравнивание слева, колонки разделены вертикальными линиями) 
\hline % Горизонтальная линия
{\bf Заголовок 1} & {\bf Заголовок 2} \\ \hline
ячейка 1/1 & ячейка 1/2 \\ \hline
ячейка 2/1 & ячейка 2/2 \\ \hline
ячейка 3/1 & ячейка 3/2 \\ \hline
\end{tabular}
\end{table}
  
\newpage

\subsection{Дополнительные возможности}\label{sec:table:additional}
\subsubsection{Пакет {\tt multirow}}\label{sec:table:multirow}

Пакет {\tt multirow} (см.~\ref{sec:work:package}) позволяет объединять ячейки таблицы. В аргументе команды {\tt \textbackslash\/multirow} задается, сколько ячеек нужно объединить, ширина конечной ячейки и непосредственно текст ячейки.

Команда {\tt \textbackslash\/cline\{\ldots\}} позволяет прочертить горизонтальную линию не по всей ширине таблицы, а охватывая заданное число колонок. Номера первой и последней из колонок, которые необходимо соединить горизонтальной линией, указываются в фигурных скобках.

Ниже приведен исходный код для создания таблицы с объединенными ячейками.

\noindent\texttt{\textbackslash\/begin\{tabular\}\{|c|c|c|\} \\
   \textbackslash\/hline \\
   \textbackslash\/multirow\{2\}\{3cm\}\{\textbackslash\/hspace\{0.25cm\} Заголовок\}  \\
    \& \textbackslash\/multicolumn\{2\}\{c|\}\{Общий заголовок\}\textbackslash\/\textbackslash \, \textbackslash\/cline\{2-3\} \\
    \&  столбец 1 	\&  столбец 2  \textbackslash\/\textbackslash \, \textbackslash\/hline \\
строка 1 \&	ячейка 1/1 \&	ячейка 1/2 \textbackslash\/\textbackslash \, \textbackslash\/hline \\
строка 2 \&	ячейка 2/1 \&	ячейка 2/2 \textbackslash\/\textbackslash \, \textbackslash\/hline \\
     \textbackslash\/end\{tabular\}}

Результатом компиляции данного кода будет таблица:

\begin{tabular}{|c|c|c|}
   \hline
   \multirow{2}{3cm}{\hspace{0.25cm} Заголовок} 
    & \multicolumn{2}{c|}{Общий заголовок }\\ \cline{2-3}
    &  столбец 1 	&  столбец 2  \\ \hline
строка 1 &	ячейка 1/1 &	ячейка 1/2 \\ \hline
строка 2 &	ячейка 2/1 &	ячейка 2/2 \\ \hline
     \end{tabular}


\subsubsection{Пакет {\tt array}}\label{sec:table:array}

При подключении пакета {\tt array} (см.~\ref{sec:work:package}) появляется простой способ борьбы с примыканием горизонтальных линеек к тексту: надо присвоить ненулевое значение параметру {\tt \textbackslash\/extrarowheight}. Это --- величина, которая добавляется к высоте каждой строки таблицы, по умолчанию равная нулю. Для лучшего отделения линеек от текста хорошо присвоить ей значение 2--3 пункта:

{\tt \textbackslash\/setlength\{\textbackslash\/extrarowheight\}\{3pt\}}

Также при подключении пакета {\tt array} наряду с выражением {\tt p\{\ldots\}}, можно использовать выражения {\tt m\{\ldots\}} и {\tt b\{\ldots\}}. Как и {\tt p\{\ldots\}}, они указывают, что в колонке находится абзац текста ширины, заданной в фигурных скобках. Отличается способ выравнивания:\\
{\tt p\{\ldots\}} --- выравнивание по верхней строке абзаца;\\
{\tt b\{\ldots\}} --- выравнивание по нижней строке абзаца;\\
{\tt m\{\ldots\}} --- абзац выравнивается по середине своей высоты.


\subsubsection{Пакет {\tt makecell}}\label{sec:table:makecell}

Для создания многострочных ячеек в таблицах удобно использовать команду {\tt \textbackslash\/makecell}. Предварительно в преамбуле необходимо подключить пакет {\tt makecell} (см.~\ref{sec:work:package}).


\subsection{Многостраничные таблицы}\label{sec:table:longtable}

\begin{sloppypar}
Для создания многостраничной таблицы необходимо подключить пакет {\tt longtable} (см.~\ref{sec:work:package}). Ниже находится пример исходного кода для размещения многостраничной таблицы:
\end{sloppypar}

\noindent\texttt{\textbackslash\/begin\{longtable\}[htb]\{|m\{5cm\}|m\{5cm\}|\}  \textcolor{red}{\% Ширина колонок 5 см}\\
\textbackslash\/captionsetup\{singlelinecheck=off\}\\
\textbackslash\/caption*\{\textbackslash\/hbox to 3.7in \{\textbackslash\/tablename \textbackslash;\textbackslash\/thetable\{\} -\/-\/- Название таблицы\}\} \\
\textbackslash\/label\{tab:label\}\textbackslash\/\textbackslash  \textcolor{red}{\% Метка таблицы} \\
\textbackslash\/hline  \textcolor{red}{\% Горизонтальная линия} \\
\{Заголовок 1\} \& \{Заголовок 2\}  \textbackslash\/\textbackslash \\
\textbackslash\/hline \\
\textbackslash\/endfirsthead  \textcolor{red}{\% Конец заголовка на первой странице} \\
\textbackslash\/caption*\{\textbackslash\/hbox to 3.8in \{Продолжение таблицы \textbackslash\/thetable\{\} \textbackslash\/hfill \}\}\textbackslash\/\textbackslash \\
\textbackslash\/hline \\
\{Заголовок 1\} \& \{Заголовок 2\} \textbackslash\/\textbackslash  \\
\textbackslash\/hline \\
\textbackslash\/endhead  \textcolor{red}{\% Конец заголовка на последующих страницах} \\
ячейка 1/1 \& ячейка 1/2 \textbackslash\/\textbackslash \, \textbackslash\/hline \\
ячейка 2/1 \& ячейка 2/2 \textbackslash\/\textbackslash \, \textbackslash\/hline \\
ячейка 3/1 \& ячейка 3/2 \textbackslash\/\textbackslash \, \textbackslash\/hline \\
\ldots \\
ячейка n/1 \& ячейка n/2 \textbackslash\/\textbackslash \, \textbackslash\/hline \\
\textbackslash\/end\{longtable\}}

По умолчанию многостраничные таблицы размещаются по центру. Для выравнивания по левому краю необходимо до начала таблицы добавить команду

{\tt \textbackslash\/LTleft=0pt}










\section{Макет документа}\label{sec:layout} 

Формат листа можно задать в первой команде документа, например:

{\tt \textbackslash\/documentclass[a4paper, times new roman, oneside]\{article\}}

где {\tt a4paper} --- установка размера листа бумаги {\tt A4} (по умолчанию {\tt letter});
{\tt oneside} --- форматирование документа для односторонней печати.

Для управления размерами макета через выставку полей необходимо загрузить пакет {\tt vmargin} и указать в преамбуле значения полей:

\begin{sloppypar}
{\tt \textbackslash\/setmarginsrb\{левое поле\}\{верхнее поле\}\{правое поле\}\{нижнее поле\}\{высота верхнего колонтитула\}\{расстояние от текста до верхнего колонтитула\}\{высота нижнего колонтитула\}\{расстояние от текста до нижнего колонтитула\}}
\end{sloppypar}

Конкретные значения аргументов команды {\tt \textbackslash\/setmarginsrb} могут быть следующими:

{\tt \textbackslash\/setmarginsrb\{20mm\}\{25mm\}\{10mm\}\{15mm\}\{0pt\}\{0mm\}\{14pt\}\{10mm\}}

Команда {\tt \textbackslash\/pagestyle\{\}} задает стиль страницы:

\noindent{\tt empty} --- страница выводится без каких-либо колонтитулов --- только текст; \\
{\tt plain} --- выводится только номер страницы в нижнем колонтитуле; \\
{\tt headings} --- в верхнем колонтитуле выводится номер страницы и информация, определяемая классом документа.








\section{Титульный лист}\label{sec:title} 

Титульный лист печатается при помощи команды {\tt \textbackslash\/maketitle}. Параметры титульного листа задаются в преамбуле:

\noindent{\tt \textbackslash\/title\{Название\}} \\
{\tt \textbackslash\/author\{Автор\}} \\
{\tt \textbackslash\/date\{Дата\}}

\begin{sloppypar}
Чтобы очистить стиль титульной страницы, используется команда {\tt \textbackslash\/thispagestyle\{empty\}}.
\end{sloppypar}







\section{Конвертация {\tt tex} файла в другие форматы}\label{sec:convert} 
 
Чтобы конвертировать файл формата {\tt tex} в формат {\tt html} (в кодировке {\tt utf8}), нужно:

\begin{enumerate}
\item скопировать {\tt filename.tex} в папку {\tt C:\textbackslash\/Users\textbackslash\/Username};
\item скопировать рисунки в папку {\tt C:\textbackslash\/Users\textbackslash\/Username\textbackslash\/Изображения};
\item набрать в командной строке 

{\tt make4ht -u filename.tex}

           или
 
{\tt htlatex filename.tex}

\end{enumerate}

  



Чтобы создать файл с расширением {\tt odt}, нужно:

\begin{enumerate}
\item скопировать {\tt filename.tex} в папку {\tt C:\textbackslash\/Users\textbackslash\/Username};
\item скопировать картинки в папку {\tt C:\textbackslash\/Users\textbackslash\/Username\textbackslash\/Изображения};
\item набрать в командной строке 

{\tt make4ht -f odt filename.tex}

\end{enumerate}

   
   
   
   
   



\appendix % Приложение
\newpage                                                                                    
\intro{Перечень принятых обозначений}\label{symbol}

В тексте документа используются следующие соглашения:
\begin{itemize}
\renewcommand{\labelitemi}{\cdash--*}
    \item команды и вывод в печатную версию приведены \texttt{моноширинным} шрифтом (символы имеют равную ширину);
    \item текст после символа \textcolor{red}{\tt \%} в примерах исходного кода означает комментарий и выделен \textcolor{red}{красным} цветом.
\end{itemize}
                                                                                 
\newpage
\intro{Перечень команд для задания символов}\label{binary}

\begin{sloppypar}
Для получения необходимого символа необходимо задающую его команду поместить в процедуру форматирования математических операций (например, \texttt{\$~\ldots~\;\$}).
\end{sloppypar}

\vspace{10pt}

\LTleft=0pt
\begin{longtable}[htb]{m{3cm}<\centering m{1cm}  m{5cm}<\centering}
{\bf Символ} && {\bf Команда}  \\ % Заголовок 
\endfirsthead
\caption*{\hbox to 3.7in {Продолжение перечня команд для задания символов}}\\ 
{\bf Символ} && {\bf Команда}  \\ 
\endhead
$+$ && {\tt +} \\ 
$-$ && {\tt -} \\ 
$*$ && {\tt *} \\ 
$\pm$ && {\tt \textbackslash\/pm}\\ 
$\times$ && {\tt \textbackslash\/times} \\ 
$ \div$ && {\tt \textbackslash\/div} \\ 
$\circ$ && {\tt \textbackslash\/circ} \\ 
$\bullet$ && {\tt \textbackslash\/bullet} \\ 
$\triangleright$ && {\tt \textbackslash\/triangleright} \\ 
$\triangleleft$ && {\tt \textbackslash\/triangleleft} \\ 
$\bigtriangledown$ && {\tt \textbackslash\/bigtriangledown} \\ 
$\bigtriangleup$ && {\tt \textbackslash\/bigtriangleup} \\ 
$\bigcirc$ && {\tt \textbackslash\/bigcirc} \\ 
$<$ && {\tt <} \\  
$>$ && {\tt >} \\ 
$=$ && {\tt =} \\ 
$:$ && {\tt :} \\ 
$\le$ && {\tt \textbackslash\/le} \\ 
$\ge$ && {\tt \textbackslash\/ge} \\ 
$\ne$ && {\tt \textbackslash\/ne} \\ 
$\approx$ && {\tt \textbackslash\/approx} \\ 
$\ll$ && {\tt \textbackslash\/ll} \\ 
$\gg$ && {\tt \textbackslash\/gg} \\ 
$\parallel$ && {\tt \textbackslash\/parallel} \\ 
$\to$  && {\tt \textbackslash\/to} \\
$\gets$ && {\tt \textbackslash\/gets} \\
$\uparrow$ && {\tt \textbackslash\/uparrow} \\
$\downarrow$ && {\tt \textbackslash\/downarrow} \\
$\infty$ && {\tt \textbackslash\/infty} \\
$\S$ && {\tt \textbackslash\/S}
\end{longtable}


\end{document} % Конец документа. Текст, который находится ниже, не отображается в скомпилированном файле.